Matter that lies near a gravitational lens galaxy or along the line of sight (LOS) can affect strong lensing observables at a level that is larger than modern measurement uncertainties. Using a framework for multi-plane lensing that we recently introduced \citep{McCully14}, we build three-dimensional mass configurations to quantify biases and uncertainties in parameters derived from lens models due to environmental/LOS effects. We derive a method to include voids along the line of sight, as the recovered Hubble Constant will otherwise be biased. We find that ignoring external galaxies yield poor fits and biased results. Fitting an external shear in the lens plane can account for tidal stretching from perturbing galaxies at the lens redshift or in the background, but corrections for external convergence need to be applied. The shear model cannot fully reproduce the more complicated effects from perturbing galaxies in the foreground. However, our 3-D Lens models can recover lens model parameters without bias and with scatter that is only limited by the lens profile degeneracy. We find a dramatic variation in the strength of the LOS/environment effects across different lens fields. By modeling each field individually, we produce stronger priors on the Hubble constant than by ray-tracing through N-body simulations.  We show that lens systems with large ellipticities are less sensitive to the lens profile degeneracy than more symmetric lenses, producing stronger constraints on the Hubble constant, and are therefore appealing targets for upcoming surveys like LSST.
  
  
  