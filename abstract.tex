Mass in the immediate environment of a gravitational lens galaxy or projected along the line of sight (LOS) can affect strong lensing observables like WHAT??? more than current measurement uncertainties. To quantify the resulting biases and uncertainties in lens model determined quantities like the Hubble constant $H_0$, we consider three lens models that treat the environment and sightline to the lens in different ways: the first ignores mass external to the lens (Lens-Only), the second adds an external shear to the lens plane (Lens+Shear), and the third employs our new 
framework for multi-plane lensing \citep{McCully14} to build the full three-dimensional mass configuration (3-D Lens), including the additional mass from structures in the lens environment and sightline, as well as foreground and background voids.  Not surprisingly, the Lens-Only model yields poor and biased fits.  While the Lens+Shear model can 
account for tidal stretching from perturbing galaxies at the lens redshift and in the background, it requires corrections for external convergence and cannot fully reproduce the more 
complicated effects from perturbing foreground galaxies.  Critically, our 3-D Lens models, which explicitly include convergence, recover lens model parameters without bias and with a scatter limited only by the lens profile degeneracy. Using the 3-D Lens model, we quantify the dramatic variation in the strength of the LOS and environment effects across different lens fields.  By modeling each field 
individually, we produce stronger priors on $H_0$ than by ray-tracing through N-body simulations.  We show that lens systems with high asymmetric lens configurations are less sensitive to the lens profile degeneracy, thus producing stronger constraints on $H_0$ and making appealing targets for LSST follow-up surveys.
  
  
  