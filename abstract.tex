Matter that lies near a gravitational lens galaxy or along the line of sight (LOS) can affect strong lensing observables at a level that is larger than modern measurement uncertainties. We use realistic three-dimensional mass configurations to quantify biases and uncertainties in parameters derived from lens models that treat environmental/LOS effects in different ways. Ignoring external effects altogether can yield poor fits and biased results. Fitting an external shear in the lens plane can account for tidal stretching from perturbing galaxies at the lens redshift or in the background, but corrections for external convergence need to be applied. The shear model cannot fully reproduce the more complicated effects from perturbing galaxies in the foreground. Using a framework for multi-plane lensing that we recently introduced \citep{McCully14} can recover lens model parameters without bias and with scatter that is smaller than typical measurement uncertainties. As part of the framework, we need to determine which perturbers can be treated using the multi-plane tidal approximation and which need to be treated in detail. We derive a quantity $\Delta_3 x$ that characterizes the strength of higher-order terms given a perturber's mass, projected offset, and redshift, and use it as an objective criterion for applying the tidal approximation. This quantity has dimensions of length, and we argue that it needs to be $\sim 2$ dex smaller than the uncertainties on lens image positions in order for the tidal approximation to be reasonable. Finally, we show that lens systems with small Einstein radii and large ellipticities are less sensitive to environmental/LOS effects than those that are larger or less elongated.