A second key parameter of the main lens galaxy is the Einstein radius, $R_E$ (the mass of the lens galaxy). Figure \ref{recompare} shows $\chi^2$, $h$, and the scatter in $e$ for different values of $R_E$. As $R_E$ decreases, so do the environment/LOS effects. We can understand this effect by looking at the definition of $\Delta_3 x \propto R_E^2$. If we assume that the positions of the environement/LOS galaxies are fixed, their contribution to the lensing goes down as $R_E$ decreases. The environment/LOS galaxies are more Einstein radii away from the main lens galaxy for smaller lenses. However, there is a trade-off. The lens with the smaller $R_E$ will have a shorter time delay, leading to a weaker constraint on the $h$ as described above.