Figure \ref{fig:fieldrz} illustrates $\Delta_3 x$ for the lens RXJ1131. $\Delta_3 x$ generally increases near the center, but there is no single radial cut that can be used to determine the importance of a perturber because perturber mass and redshift must also be taken into account. By following contours of constant $\Delta_3 x$, as illustrated in figure \ref{fig:fieldrz}, we can determine the effective radial offset for the perturber if it was in the main lens plane. Background perturbers are downweighted by ($1-\beta)^2$, so they must have a smaller projected offset to have the same effect as if they were in the main lens plane. A histogram of $\Delta_3 x$ is shown in figure \ref{fig:d3xhist}, showing that foreground perturbers and group members tend to have a larger $\Delta_3 x$ than background perturbing galaxies.
  
  
  
  
  