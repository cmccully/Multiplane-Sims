There are a variety of ways one could account for this bias. One possibility is to start with an empty universe $\Omega_M = 0$ and only add mass where we detect a galaxy. There is major drawback to this approach: for this to produce the correct results, the mass models must be $100\%$ complete which is unrealistic. A second, and our favored, method is to split the mass into two components: a smooth mass background and a series of discrete mass components (i.e. galaxies) along the LOS. The next question becomes how to include the smooth mass component in our framework. The simplest mass plane in our framework includes only convergence, measuring the deviation from the mean mass density of the universe. One imagine using a finely spaced grid of redshift planes, each with some positive or negative convergence. In the planes where we detect a galaxy, the convergence is positive (as we have included an SIS for example) while in the empty planes, we add a negative convergence to account for the empty space. But how many redshift planes do we need? Presumably, the more planes, the better. This is suggestive of an integral, which we will now examine more quantitatively.
  
  