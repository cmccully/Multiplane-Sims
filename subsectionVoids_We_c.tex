\subsection{Including Empty Space}
In traditional single plane lensing, it reasonable to approximate that the lens galaxy lies on top of a smooth background and that the lens galaxy does not significantly contribute to the overall geometry of the universe. However, as we begin to include more and more galaxies along the line sight, the cosmological effects become non-negligible.


We can consider voids as the sum of negative mass planes relative to the mean density of the universe. We start with a thin slab that extends from proper distance $r_p$ to $r_p+dr_p$.  The density contrast within this slab is $\drho_p = \rho_p - \bar\rho_p$, where $\rho_p = \rho_0 (1+z)^3$ is the proper density.  We can associate this slab with a plane of convergence $d\kappa$ as follows.  Consider a volume element that corresponds to solid angle $d\Omega$ and thickness $dr_p$.  The proper volume is
  