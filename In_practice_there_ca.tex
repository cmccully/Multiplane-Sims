In practice, there can be several hundred galaxies projected within $\sim5$ arcminutes of the main lens galaxy. We could use the full multi-plane lensing formalism \citep{PLW}, but the multi-plane lens equation is recursive, making this approach computationally infeasible.
In our previous paper (\citealp{McCully14}, see \citealt{Schneider14} for an alternate derivation.), we presented a hybrid framework for multi-plane gravitational lensing that fills the gap between using the full multi-plane lens equation (which can be computationally expensive) and treating everything as a simple external shear (which omits higher-order effects that can be significant for objects projected near the lens). The framework can handle any mixture of ``main'' planes (strong lenses) that are given full treatment and ``tidal'' planes (weak lenses) that are treated using the tidal approximation. Our framework includes non-linear redshift effects, but is also computationally efficient because we employ the tidal approximation for most of the LOS galaxies.
  
  