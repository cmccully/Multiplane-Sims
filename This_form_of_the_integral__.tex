This form of the integral is recursive and is therefore unwieldy to use in practice, but is also suggestive that $\betahat$ can be described by a differential equation. We are going to take derivatives of \ref{eqn:betahatint} with respect to comoving distances, so it is first useful to examine derivatives of $\beta_{i j}$. In terms or comoving distances, $X_i$, 
\begin{eqnarray}
\beta_{i j} &=&\frac{D_{i j} D_s}{D_{i s} D_j} = \frac{(X_j - X_i) X_s}{X_j (X_s - X_i)} \\
\frac{d \beta_{i j}}{d X_j} &=& \frac{X_i X_s}{X_j^2 (X_s - X_i)}\\
\label{eqn:d2betahatdbetahat} 
\frac{d^2 \beta_{i j}}{d X_j} &=& -2 \frac{X_i X_s}{X_j^3 (X_s - X_i)} = -\frac{2}{X_j} \frac{d \beta_{i j}}{d X_j} 
\end{eqnarray}
  
  
  
  
  
  
  
  
  