\begin{equation}
\x_s = (\I - \GammaMat) \x_1 - \al_g\left((\I - \beta\GammaMat) \x_1 + \mathcal{O}(\x_1^3)\right) + \mathcal{O}(\x_1^3).
\end{equation}
This looks similar to equation \ref{eqn:background}, but there is a key difference: instead of just having a multiplicative effect on the source position like the background perturber, the deflection from the foreground perturber enters the lens equation \textit{inside the argument} of the deflection of the main galaxy. For a foreground perturber, there is no $\GammaMat_{\rm eff}$ that we can define as with the background perturber because $\GammaMat$ is inside the argument of $\al_g$; thus the lens equation is not linear in $\GammaMat$ (see also M14). In principle, one can define a scaled coordinate based on the argument of the deflection to transform this equation to look like the standard lens equation \citep[e.g.,][]{Schneider97,Keeton03}. However, this requires care, as the new quantities do not correspond to the \textit{observed} image positions that are typically used in lens modeling. Another alternative is to rescale the mass of the main lens galaxy to account for the change in the argument of the deflection \citep{Schneider97}, but this also requires care because the mass predicted by the lens models is no longer the true physical mass of the lens.
  
  
  