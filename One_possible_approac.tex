One possible approach to using gravitational lensing for precision cosmology is to search for one or a few ``golden lenses'' to use to measure cosmological parameters. One consideration when choosing these lenses needs to be the environment. We find that there is significant diversity in lens LOS/environments, with both the number and strengths of the perturbing galaxies vary from field to field.  We also examine which lens properties affect the sensitivity to the environment/LOS. We find that for a highly elliptical main lens galaxies produce a longer time delay, giving a stronger constraint on $h$ assuming a constant uncertainty of 1 day on the time delay. Highly elliptical systems also have a strong second order contribution which can limit the effects of the shear from the environment/LOS. Main lens galaxies with small Einstein radii are less sensitive to LOS/environment effects than those with large Einstein radii because given a set of galaxies with fixed angular positions, the perturbers are more Einstein radii away for the smaller lens galaxy than for the larger.