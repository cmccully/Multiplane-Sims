Regardless of the ellipticity, the Shear models overpredict the Hubble Constant. This effect is due to the mass sheet degeneracy, as the Shear models do not include convergence. Remarkably, as the $e$ increases, the scatter in the recovered $h$ values decreases. Part of this effect is due to the relative strength of the environment/LOS and internal quadrapoles as discussed above. However, there is also a second effect here. The relative precision that can be achieved in measurements of the Hubble Constant is directly proportional to the fractional uncertainty in the time delay measurement. Therefore, if the fractional uncertainties in the time delays for two different lenses are the same, their precision in $h$ will also be the same. However, as stated above, we do not assume a constant fractional uncertainty on the time delay, but an absolute uncertainty of one day, because the accuracy of the time delay is more related to the cadence of the observations than the length of the time delay. For highly elliptical, asymmetric lens configurations, the time delays are longer, as illustrated in Figure \ref{timedelays} \citep[see also][]{Witt00}. Assuming an absolute time delay uncertainty, longer time delays will have a smaller fractional uncertainty, leading to stronger constraints on the Hubble Constant.