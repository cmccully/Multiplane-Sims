To test the prediction that galaxies in the background are down-weighted compared to those in the foreground, we ran a set of simulations griding over projected offset and redshift. For these simulations we fixed the mass of the perturber to be $10^{12} M_\odot$. We show the scatter in the Hubble Constant from the fits for the 3-D Lens models in Figure \ref{toyhd3x}. \footnote{When plotting the results, we use an $\asinh$ color scaling to illustrate the dynamic range of the variation. At large values, $\asinh$ acts like a logarithm, but at small values, it becomes linear. The $\asinh$ is well defined for both positive and negative numbers and is conveniently antisymmetric.} This scatter is driven by perturbations due to the lens model rather than an ``external convergence''. We find that perturbers behind the lens must be closer in projection than perturbers in the lens plane to have the same effect on the scatter in $h$. We have plotted a contour of constant flexion terms, $\Delta_3 x$, which closely traces contours of constant scatter in the Hubble Constant, $h$. In the background, the effects of the point mass perturber are weaker because of the $\beta$ factors in $\Delta_3 x$. In the foreground, the strength of the perturber increases as $z \rightarrow 0$ due to its increasing Einstein radius as $z \rightarrow 0$. Both as the perturbing galaxy gets close in projection to the main lens and as its redshift goes to zero, the point mass approximation breaks down. However, as we are interested in the quantity to prescribe when we must treat The shape of this curve is similar to the ``suspension bridge'' shape found by \citet{Momcheva06}.