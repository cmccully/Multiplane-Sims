One of the key issues is that lens galaxies are not isolated systems. Matter in the environment of the lens galaxy and mass projected along the line of sight (hereafter ENV/LOS) can produce perturbations in the lensing potential that cannot be ignored as we enter the era of ``Precision Lensing'' (typically defined by the goal of constraining the Hubble constant to $1\%$). To lowest order, using the ``tidal approximation'', perturbing galaxies contribute ``external convergence'', which produces extra (de-)magnification, and ``external shear'', which distorts the images of the source, transforming circles to ellipses. The ``external convergence'' is one of the dominant components of the uncertainty budget on measurements of the Hubble Constant \citep{Suyu12}. So, our goal for this work is to explore the effects due mass outside the main lens galaxy on the measured Cosmology \footnote{Here we focus on the Hubble Constant, but similar arguments apply for time-delay cosmography measurements that have been proposed to be used to measure dark energy \citep{Treu13}.}. To quantify these effects, we employ two of the most common methodologies that treat the lensing LOS/environment.

