External convergence and shear arise from two components: the lens environment and the line of sight (hereafter environment/LOS; e.g.,\ \citealt{Jaroszynski14,Seljak94,Bar-Kana96,Keeton97}). Many lens systems have significant contributions from in their local environment. For example, HE0435 has a neighbor \citep{Kochanek06}, MG0414 \citep{Tonry99}, RXJ1131 \citep{Sluse03}, and B2114 \citep{King99} all have satellites that are presumably close enough to matter. Using galaxy demographics, \citet{Keeton00} estimate that at least $25\%$ of lens galaxies are part of groups or clusters, with several spectroscopically confirmed to lie in groups or clusters \citep[][and references therein]{Momcheva06}. \citet{Keeton04} show that ignoring neighbor galaxies, like group members, can lead to a bias in the fitted  parameters corresponding to a systematic uncertainty in modeling the lens system. Some models do treat nearby perturbers exactly \citep[e.g.][]{Fadely12}, but typically the decision whether to include a neighbor galaxy in a model is \textit{ad hoc}.