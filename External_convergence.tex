One key issue is that lens galaxies are not isolated systems. The ``external convergence'' produced by the environment of the lens galaxy and mass projected along the line of sight (hereafter environment/LOS) is one of the dominant components of the uncertainty budget on measurements of the Hubble Constant \citep{Suyu12}. So, our goal for this work is to quantify the effects due mass outside the main lens galaxy on the measured Cosmology. Here we focus on the Hubble Constant, but similar arguments apply for time-delay cosmography measurements that have been proposed to be used to measure dark energy \citet{Treu13}. To this end, we aim to answer the following questions: which individual galaxies produce the strongest perturbations, and how should they be modeled? Which lenses show a strong contribution from their local environment? Which image configurations produce the tightest constraints on $H_0$?
