The first step of building 3-D lens models is to convert our photometric and spectroscopic observations into mass models. We construct the 3-D mass models using an updated version of the methodology in \citep{Wong11}, which we briefly summarize here. We start by assigning redshifts to the LOS galaxies that lack spectroscopy. If the field is in the SDSS footprint, we assign the photometric redshift to the galaxy. If not, we draw from a smoothed version of the spectroscopic redshift distribution in that particular field. We model the galaxies as singular isothermal spheres with velocity dispersions and truncation radii (with the associated scatter) taken from scaling relations derived from a galaxy-galaxy weak lensing analysis in the CFHTLS by \citep{Brimioulle13}. For lenses that lie in galaxy groups or clusters, we treat the group member galaxies in a slightly different way. We calculate a total dynamical mass for the group \citep[][; Wilson et al. in preparation]{Girardi98,Momcheva06} and apportion the mass between the group dark matter halo and the group galaxies. The dark matter halo is assumed to be a spherical NFW profile. The halo concentration is taken from the mass-concentration relation from simulations by \citep{Zhao09}, with an assumed scatter of 0.14 dex \citep{Bullock01}. The fraction of the total group mass assigned to the halo, $f_{\rm{halo}}$, is randomly drawn from a uniform distribution such that the group galaxies are not truncated below twice their effective radii, nor beyond their $R_{200}$ (the radius at which the mean density enclosed is 200 times the matter density at that redshift). The remainder of the mass, $M_{\rm{group}}∗(1−f_{\rm{halo}})$, is assigned to the group galaxies, which are scaled such that they have the same density at their truncation radii. All group galaxies and the group halo are assumed to be in the same redshift plane as the lens, even though there may be slight redshift differences due to peculiar velocities.
