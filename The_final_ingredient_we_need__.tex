The final ingredient we need for our LOS framework is the mass completeness. To construct this, we take the average mass per redshift bin across all of of fields for which we have constructed mass models. We then fit a smooth function to the completeness of the form $f(z) = e^{-z^a / b}$. We find best fit values of $a = 3.23$ and $b = 0.183$. We then use this mass completeness function to account for voids across all of our beams. Because we have used the average across all of our beams instead of fitting each beam individually, this allows for individual beams to be over dense or under dense compared to the mean density of the universe. The only assumption is that the mean of our total sample is approximately the mean density of the universe.  
  