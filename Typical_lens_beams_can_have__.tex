Typical lens beams can have hundreds of galaxies along the LOS, making the analytic form of the lens equation intractable. Instead, we turn to numerical simulation to test the LOS/Environment effects on lensing. Our procedure is to use Lensmodel \citep{Keeton01} to simulate mock lenses making no approximations and then fit them using a variety of models. The parameters in the simulations are allowed to vary to find a ``better fit'' even if the fit parameters are farther away from the truth. In our Monte Carlo simulations, we use the full recursive lens equation given by equation \ref{eqn:mp_full}, making no approximations to generate the mock data, then comparing to different treatment of the LOS/Environment (described below). To make quantitative comparisons, we define a $\chi^2$ of the fit, using fiducial measurement uncertainties of 3 milliarcseconds on the positions of the main galaxy and each of the images, $5\%$ on the flux, and 1 day on the time delay. These values are chosen to represent lenses targeted with some of the best current instruments like \textit{HST} and monitoring campaigns like COSMOGRAIL \citep{Eigenbrod05}. We do not explicitly add scatter to the mock observed quantities here, although preliminary tests that include such scatter adds a ``floor'' in the $\chi^2$ and in the scatter in the recovered lens parameters. Otherwise these results are qualitatively similar, so we do not present them here.