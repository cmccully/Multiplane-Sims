The first method accounts for the LOS/environment by treating the magnitude and direction of the external shear as free parameters to be optimized for individual lens systems. This widely-used approach has a few possible limitations. First, this approach neglects higher-order effects beyond shear, which may be significant for objects sufficiently close to the optical axis. This approach also neglects non-linear effects that arise from having mass in multiple redshift planes \citep[][]{McCully14,Jaroszynski12}. Also, the lens models cannot be used to directly constrain the external convergence because of the mass sheet degeneracy \citep{Falco85}.  To avoid biases in the derived cosmological parameters, external convergence must be added back in after the initial fit using some outside constraint such as weak lensing \citep{Nakajima09, Fadely10} or the number density of galaxies near the lens \citep{Suyu10, Suyu13, Collett13}, which is calibrated by ray tracing through cosmological simulations \citep[e.g.,][]{Hilbert09}. \citet{Schneider13} have argued against this approach as it probes much larger scales than those relevant for strong lensing. The calibration for the external convergence is also derived from statistical studies that may have limited applicability to individual lens environments \citep{Wong11}.
  