In our Monte Carlo simulations, we use the full recursive lens equation given by \refeq{mp_full}, making no approximations to generate the mock data. We then fit the mock data using three different models for the environment/LOS, described below. We define our fiducial measurement uncertainties as 3 milliarcseconds in the positions of the main galaxy and each of the images, $5\%$ in the flux, and 1 day in the time delay. These values are chosen to represent lenses targeted with some of the best current instruments like \textit{HST} and monitoring campaigns like COSMOGRAIL \citep{Eigenbrod05}. We use these uncertainties to define a $\chi^2$ of the fit. We do not explicitly add scatter to the mock observed quantities here, although preliminary tests that include such scatter has a ``floor'' in the $\chi^2$ and in the scatter in the recovered lens parameters. Otherwise these results are qualitatively similar, so we do not present them here. We will revisit this question in a forthcoming paper (Wong et al.\ in prep.).