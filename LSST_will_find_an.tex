LSST will find an immense number of new strong lens systems. There are a variety of strategies on how to use this upcoming dataset for cosmology. One possibility is to use all of the lenses to beat down the uncertainties using statistics. However, if we are entering the systematics-dominated regime, this approach will be limited unless one can account for the systematic uncertainties (e.g., with our LOS framework). Another strategy is to use the large number of lenses discovered by LSST to search for a few rare, ``golden'' lenses, whose systematic uncertainties are small. One possible criterion for a ``golden'' lens could be to have a have weaker environment/LOS effects (see Figure \ref{fig:allfields}). Based on our analysis above, to minimize environment/LOS effects, we search for lenses with a large $R_E$ and high $e$. Assuming an absolute uncertainty of a day (which may or may not be a good assumption) suggests that our strongest constraints on the Hubble Constant will come from systems with long time delays, like those from highly elliptical, asymmetric systems. The strong quadrapole due to the ellipticity would also dominate the quadrapole from the environment/LOS, further reducing the contribution of environment/LOS effects to lensing observables. Highly asymmetric lenses also appear to be less sensitive to the lens profile degeneracy. While suggestive, these results merit further investigation to get the most out of future surveys like LSST.
  