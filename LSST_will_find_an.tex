LSST will find an immense number of new strong lens systems. There are a variety of strategies on how to use this upcoming dataset for cosmology. One possibility is to use all of the lenses to beat down the uncertainties using statistics. However, if we are entering the systematics-dominated regime, this approach will be limited unless one can account for the systematic uncertainties (e.g., with our 3-D Lens models). Another strategy is to use the large number of lenses discovered by LSST to search for a few rare, ``golden'' lenses, that have small systematic uncertainties. One possible criterion for a ``golden'' lens could be to have a have weaker contribution from the environment/LOS (see Figure \ref{fig:allfields}). Based on our analysis above, to minimize environment/LOS effects, we should search for lenses with a large $R_E$ and high $e$. These large,  asymmetric lenses are less sensitive to the lens profile degeneracy. While suggestive, these results merit further investigation to get the most out of future surveys like LSST.
  