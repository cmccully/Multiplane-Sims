In the previous section, we presented analytic arguments about how we expect our new LOS framework to behave. Specifically, we defined the quantity $\Delta_3 x$ to characterize the strength of the higher order terms for a given galaxy and derived a method to account for the empty space between galaxies. We now shift our focus to numerical simulations using our LOS framework. These results are complementary to the analytic results above, as the numerical simulations can capture significantly more complicated behavior. We use Lensmodel \citep{Keeton01} to simulate mock lenses and then fit them using various models. The parameters in the simulations are allowed to vary to find a ``better fit'' even if the fit parameters are farther away from the truth. We begin by examining ``toy models'' that contain a single perturbing galaxy. Our goal is to test if the scatter in the fitted parameters is related to the perturbations in the image positions given by $\Delta_3 x$. In these simulations, we test the different terms in $\Delta_3 x$ separately, isolating their effects. We then study more realistic environment/LOS models that include $\sim 300$ galaxies, testing how different cuts on $\Delta_3 x$ affect their recovered lens parameters.
  