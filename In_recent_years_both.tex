In recent years, both the quantity and quality of strong lens data have improved. There are now roughly a hundred known lensed quasars \citep[e.g.,\ \fnurl{CASTLeS}{http://www.cfa.harvard.edu/castles/};][]{SQLS,CLASS}(e.g.,\ \fnurl{CASTLeS}{http://www.cfa.harvard.edu/castles/}; \citealt{SQLS,CLASS}) and a comparable number of strongly lensed galaxies \citep[e.g.,][]{Bolton08,Cassowary}. These samples will increase dramatically in the near future with LSST \citep[e.g.,][]{LSST, Coe09, Oguri10, Collett15}. Here, we focus on quasar lenses because the compact source can vary rapidly enough to enable measurements of lens time delays (though much of this discussion also applies to strongly lensed supernovae \citealt{Kelly15}). The relative positions and fluxes of quasar images are routinely measured to high precision using the \textit{Hubble Space Telescope} \citep[e.g,.][and references therein; CASTLeS Collaboration]{Lehar00,Sluse12}. Our understanding of gravitational lenses and the constraints they place on cosmology is no longer limited by observations, but by systematic uncertainties. One key issue is that lens galaxies are not isolated systems. The largest component of the uncertainty budget for measuring the Hubble Constant with lensing is associated with external convergence \citep{Suyu12}.
  