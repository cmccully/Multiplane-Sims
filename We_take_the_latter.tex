We take the latter approach. To analyze realistic environment/LOS mass models, we use our extensive spectroscopic and photometric data on several lens systems \citep{Wong11}. We follow the method of \citet{Wong11} to convert the observations into lensing mass models that include galaxies in any group surrounding the lens, a common dark matter halo for that group, and all other individual galaxies along the LOS. The group halo is assumed to be an NFW profile \citep{Navarro96}, and the mass is derived from the group velocity dispersion. The galaxies are modeled as truncated singular isothermal spheres with mass assigned using the Faber-Jackson relation \citep{Faber76}. For this work, we focus on a single realization from the range of allowed possibilities.  We are less focused on reproducing actual beams in the universe than on using plausible mass models to use to validate our methodology, calibrate the shear approximation, and test the fitted parameters for biases and scatter due to LOS/environment effects.